%%%%%%%%%%%%%%%%%%%%%%%%%%%%%%%%%%%%%%%%%%%%%%%%%%%%%%%%%%%%%%%%%%%%
%% %%	Posterdown PDF class for LaTeX files	 08-JAN-2019
%% %%	For any information please send an e-mail to:
%% %%		brentthonre18@gmail.com (Brent Thorne)
%% %%
%% %%	Initial class provided by:
%% %%		Brent Thorne
%% %% Contributors: Shea Connell (SC)
%%%%%%%%%%%%%%%%%%%%%%%%%%%%%%%%%%%%%%%%%%%%%%%%%%%%%%%%%%%%%%%%%%%%

\documentclass[article,30pt,extrafontsizes]{memoir}

%utf-8 seems to be important
\RequirePackage[utf8]{inputenc}
\RequirePackage[T1]{fontenc}
\RequirePackage{lmodern}
\RequirePackage{multicol}
\RequirePackage{graphicx}
\RequirePackage{lipsum}
\RequirePackage{blindtext}
\RequirePackage[svgnames,table]{xcolor}
\RequirePackage{tikz}
\RequirePackage[framemethod=tikz]{mdframed}
\RequirePackage{color}
\RequirePackage{geometry}
\RequirePackage{adjmulticol}
\RequirePackage[skins,most,listings,skins]{tcolorbox}

%For kable extra package :)
\RequirePackage{booktabs}
\RequirePackage{longtable}
\RequirePackage{array}
\RequirePackage{multirow}
\RequirePackage{wrapfig}
\RequirePackage{float}
\RequirePackage{colortbl}
\RequirePackage{pdflscape}
\RequirePackage{pagecolor}
\RequirePackage{tabu}
\RequirePackage{threeparttable}
\RequirePackage{threeparttablex}
\RequirePackage[normalem]{ulem}
\RequirePackage{makecell}
\RequirePackage{wrapfig}

%rof hyperrefs
\RequirePackage{hyperref}
\hypersetup{
    colorlinks=true,
    linkcolor=linkcol,
    citecolor=citecol,
    filecolor=linkcol,
    urlcolor=urlcol,
}
%For figure and table placement
\RequirePackage{float}
\floatplacement{figure}{H}
\floatplacement{table}{H}

%%%%%%%%% COLOURS %%%%%%%%
%Fill/ Line Colours
\definecolor{titleboxbgcol}{HTML}{45ADA8}
\definecolor{titleboxbordercol}{HTML}{547980}
\definecolor{columnlinecol}{HTML}{008080}
\definecolor{bodybgcol}{HTML}{ffffff}
\definecolor{sectitlebgcol}{HTML}{547980}
\definecolor{sectitlebordercol}{HTML}{547980}
% Text Colours
\definecolor{titletextcol}{HTML}{FFFFFF}
\definecolor{authortextcol}{HTML}{FFFFFF}
\definecolor{affiliationtextcol}{HTML}{FFFFFF}
\definecolor{sectitletextcol}{HTML}{ffffff}
\definecolor{bodytextcol}{HTML}{594F4F}
\definecolor{footnotetextcol}{HTML}{ffffff}
\definecolor{citecol}{HTML}{CC0000}
\definecolor{urlcol}{HTML}{008080}
\definecolor{linkcol}{HTML}{008080}


%Memoir spacing options
%spacing between figure/ table and caption
\setlength{\abovecaptionskip}{0.4in}
\setlength{\belowcaptionskip}{0.2in}
\captionnamefont{\footnotesize\sffamily\bfseries}
\captiontitlefont{\footnotesize\sffamily}

%define column options
\setlength{\columnseprule}{0pt}
\def\columnseprulecolor{\color{columnlinecol}}

%define section title features
\setsubsubsecheadstyle{\small\color{sectitletextcol}\textbf}% Set \section style
\setsecnumformat{}
\def\sectionmark#1{\markboth{#1}{#1}}

%%%%%%%%%%%% TCOLORBOXES TO THE RESCUE %%%%%%%%%%%%%%%%%%%%
%Title Box
\newtcolorbox{topbox}{
enhanced,
colback=titleboxbgcol,
colframe=titleboxbordercol,
halign=center,
boxrule=1cm,
sharp corners=all,
 overlay={
    \node[anchor=south west]
      at ([xshift=1in,yshift=1in]frame.south west)
       {\includegraphics[width=5in]{Figures/SHESC}};
    \node[anchor=south east]
      at ([xshift=-1in,yshift=1in]frame.south east)
       {\includegraphics[width=5in]{Figures/SHESC}};}

}
%Body Section Title Box
\newtcolorbox{myboxstuff}[1][]{
code={\parindent=0em},
colframe=sectitlebordercol,
nobeforeafter,
left skip=0pt,
valign=center,
halign=center,
fontupper=\Large\bfseries,
colupper=sectitletextcol,
boxrule=2mm,
colback=sectitlebgcol,
sharp corners=uphill, #1}
\newcommand{\mybox}[1]{%
\begin{myboxstuff}
\strut #1
\end{myboxstuff}%
}
\makeheadstyles{MyBox}{
    \setsecheadstyle{\mybox}
}
\headstyles{MyBox}\makepagestyle{MyBox}
%-----------------------------------------------------
%Make sure that the page is empty of any preset items from memoir
\thispagestyle{empty}

%biblatex options
\RequirePackage[sorting=none,backend=biber]{biblatex}
\renewcommand*{\bibfont}{\small} %% SC
\bibliography{MyLibrary}
\defbibheading{bibliography}[\bibname]{%
\setlength\bibitemsep{0.8\itemsep} %% SC
\section*{#1}%
\markboth{#1}{#1}}
\AtBeginDocument{%
  \renewcommand{\bibname}{References}
}

%Remove section numbering & set 2nd level header as first level
%to avoid the automatic new page generated from memoir chapter
%formatting
\counterwithout{section}{chapter}
\makechapterstyle{mydefault}{
\addtocounter{secnumdepth}{2}
\setsecheadstyle{\mybox}
\setsubsecheadstyle{\itshape}
\setsubsubsecheadstyle{\itshape}
}

%set the chapterstyle
\chapterstyle{mydefault}

%define column spacing
\setlength\columnsep{0.5in}

%spacing params
\setlength\parindent{0em}
\setlength\parskip{0em}
\setlength\hangparas{0}

%spacing after section head title
\setaftersecskip{0em}
\setbeforesecskip{1.5em}
\setlength\textfloatsep{0in}
\setlength\floatsep{0in}
\setlength\intextsep{0in}

\setstocksize{36in}{48in}
\settrimmedsize{\stockheight}{\stockwidth}{*}
\settypeblocksize{36in}{48in}{*}
\setlrmargins{*}{*}{1}
\setulmarginsandblock{2.5cm}{*}{*}
\setmarginnotes{0em}{0cm}{0cm}
\setlength{\footskip}{0cm}
\setlength{\footnotesep}{0cm}
\setlength{\headheight}{0pt}
\setlength{\headsep}{0pt}
\setlength{\trimtop}{0pt}
\setlength{\trimedge}{0pt}
\setlength{\uppermargin}{0pt}
\checkandfixthelayout

%Footnote to white
\RequirePackage{footmisc}
\def\footnotelayout{\centering\color{footnotetextcol}}

% see https://stackoverflow.com/a/47122900

% choose font family
\RequirePackage{palatino}

% define the BODYBGCOL
\newpagecolor{bodybgcol}

%sets footnote to be white hopefully
\renewcommand\footnoterule{}
\renewcommand{\thempfootnote}{\footnotesize\color{footnotetextcol}{\arabic{mpfootnote}}}

%-------------- Begin Document -------------------%
\begin{document}

%-------------- Title Box Start ------------------%
%tcolorbox allows for pictures hopefully
\begin{topbox}
  \color{titletextcol}
  \vspace{0.5in}
  \Huge{\fontfamily{phv}\selectfont Generalized additive mixed models for
archaeological network data}  \\[0.3in]  %% SC
  \color{authortextcol} \Large{Nicolas Gauthier} \\[0.2in] %% SC
  \color{affiliationtextcol} \large{School of Human Evolution and Social Change, Arizona State University} %% SC
  \vspace{1cm}
\end{topbox}
%--------------- Title Box End -------------------%
%----------------- Body Start --------------------%
% Begin body of poster
\begin{adjmulticols*}{4}{0.5in}{0.5in}
\normalsize{  %% SC
\color{bodytextcol}
\section{Introduction}\label{introduction}

Distance is a fundamental constraint on human social interaction. This
basic principle motivates the use of spatial interaction models for
estimating flows of people, information, and resources on spatial and
social networks. These models have both valid dynamical and statistical
interpretations, a key strength well supported by theory and data from
geography, economics, ecology, and genetics. To date, archaeologists
have primarily relied on the dynamical approach because the
idiosyncrasies of archaeological data make the wholesale adoption of
statistical approaches from other fields impractical. \vspace{1cm} Here,
I argue for the use of generalized additive mixed models (GAMMs) for
statistical inference on interaction networks in archaeology. GAMMs are
a flexible form of regression model well-matched to the complexities of
the archaeological record, including non-normal distributions in the
form of counts or proportions, non-independent observations with
correlated errors, and non-linear functional relationships. Using two
case studies -- an ethnographic marriage network and an archaeological
assemblage similarity network -- I illustrate how this approach can lead
to unbiased parameter estimates and more robust comparisons of competing
hypotheses. I conclude by outlining how future empirical efforts can
help to refine our thinking about past network dynamics and reveal
cross-cultural regularities of human social interaction in the present
day.

\section{Study Site}\label{study-site}

\section{Objectives}\label{objectives}

\lipsum[2]
\lipsum[4]

\section{Methods}\label{methods}

\section{Results}\label{results}

\vspace{1cm}

\vspace{2cm}

\lipsum[1-2]

\section{Next Steps}\label{next-steps}

\small\printbibliography
}
\end{adjmulticols*}
%------------------ Body End ---------------------%
%end the poster
\end{document}

